\chapter{Conclusions}\label{chap:conclusions-and-future-work}



\section*{}

This thesis proposal described the work plan for the development of a cooperative assembly system capable of learning new skills from demonstration and cooperate with human operators using an immersive augmented reality system. It's main objective is to improve the flexibility of manufacturing lines by giving the human operators intuitive teaching systems capable of improving manufacturing productivity while also reducing the time and costs of re-purposing assembly lines by giving high level cognition and cooperative skills to robots. This is a very challenging and multidisciplinary research subject with great potential for the growing presence of robotics in the manufacturing industry.

The proposed thesis builds upon the candidate's experience in 3D perception \cite{Costa2015Diss,Costa2015ICIT,Costa2015Intech,Costa2016Elsevier}, computer vision \cite{Costa2014,Costa2016ICARSC}, augmented reality\footnote{\url{https://github.com/carlosmccosta/AR-Chess}}, projection mapping and also on the industrial research experience gained during the \gls{carlos}\footnote{\url{http://carlosproject.eu/}} and \gls{clarissa}\footnote{\url{http://www.smerobotics.org/AUTOMATICA/exhibit-07-2016.html}} European research projects, and unlike previous research that focuses on specific assembly applications using traditional interfaces, the proposed system aims to provide a generic, intuitive and immersive cooperative robot assembly system capable of interacting with the operator using augmented reality while also learning the necessary assembly skills from demonstration.