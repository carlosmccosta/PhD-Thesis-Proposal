\chapter{Introduction} \label{chap:introduction}



\section*{}

FF



\section{Context}\label{sec:introduction_context}

FF



\section{Motivation and objectives}\label{sec:introduction_goals}

The main objective of this project is the development of a human-robot interaction system capable of learning from demonstration and cooperate with human co-workers in order to perform complex assembly tasks using robotic arms in industrial scenarios. The Human-Robot safety issues will be taken into consideration while planning the robot movements, but are not the main research area given that recent cooperation robots have collision detection and human safety built-in at the controller level.

To achieve the mentioned cooperation in assembly tasks, the system will need to recognize and track the assembly components using 3D perception of the environment and must also be able to track and analyze human movements in order to learn new skills or receive new commands. These learning skills will rely on cumulative knowledge and will be shared across robotic arms with different hardware configurations through the usage of a common skill / knowledge database.

To enhance the cooperation with the operator, the system will provide an augmented reality interface in order to display relevant learning / assembly information and also inform the operator of missing / damaged assembly components. Moreover, this interface provides an immersive and intuitive way of manipulating the system configuration and will ease the human-robot cooperation when the assembly tasks alternate between the operator and the robotic arm. The proposed system will be validated with the assembly of several complex objects using several types of robotic arms in order to test the learning skills along with the perception and manipulation sub- systems. These objects will be selected based on their assembly complexity and on type of demand seen in industrial scenarios.



\section{Proposal outline}\label{sec:introduction_structure}

FF
