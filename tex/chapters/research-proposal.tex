\chapter{Research proposal}\label{chap:research-proposal}



\section*{}

FF



\section{Research questions}

FF


\section{Research areas}

FF


\section{Objectives}

The main objective of this project is to devise a flexible framework for human-robot cooperative assembly capable of semantic learning by demonstration in order to improve the flexibility and efficiency of complex industrial assembly tasks by relying on the robot's precision / speed skills and on the human knowledge/dexterity. To achieve the mentioned cooperation in assembly tasks, the system will need to recognize and track the assembly components using 3D perception and must also be able to semantically analyze human movements to learn new skills or receive new commands from the operator. These learning skills will rely on \gls{cad} / \gls{sop} analysis, cumulative assembly knowledge and will be shared across robotic arms with different hardware configurations through the usage of a common skill / knowledge database.

To enhance the cooperation with the operator, the system will provide an augmented reality interface in order to display relevant learning / assembly information and also inform the operator of missing / wrong assembly components. Moreover, this interface will allow an immersive and intuitive way of inspecting and manipulating the assembly system knowledge and will ease the human-robot cooperation when the assembly tasks alternate between the operator and the robotic system.

Expected contributions include not only the formalization of generic skills, reusable in different robots and in different parts but also proposing additional human-robot strategies for improved flexibility in industrial assembly tasks.

The proposed approaches will be validated in simulation and in real hardware, with the assembly of several complex objects using different types of robotic arms in order to test the learning skills along with the perception and manipulation sub-systems. These objects will be selected based on their assembly complexity and their type of demand seen in industrial scenarios (industrial partners PSA Peugeot Citroën\footnote{\url{http://www.psa-peugeot-citroen.com/en}} and Simoldes Plásticos\footnote{\url{http://www.simoldes.com/plastics/}} have shown interest in this work proposal).



\section{Industrial applications}

FF



\section{Expected contributions}

The proposed project builds upon the candidate’s experience in 3D perception \cite{Costa2015Diss,Costa2015ICIT,Costa2015Intech,Costa2016Elsevier}, computer vision \cite{Costa2014,Costa2016ICARSC}, augmented reality\footnote{\url{https://github.com/carlosmccosta/AR-Chess}} and projection mapping, and unlike previous research that focuses on specific assembly applications using traditional interfaces, the proposed project aims to provide a generic, intuitive and immersive cooperative robot assembly system capable of interacting with the operator using augmented reality while also learning the necessary assembly skills from demonstration.
